\documentclass[12pt, a4paper]{article}

\usepackage[left=2cm, right=2cm, top=2cm, bottom=2cm]{geometry}
\usepackage[utf8]{inputenc}
\usepackage{amsmath}
\usepackage{csquotes}
\usepackage{enumerate}

\newcommand{\doctitle}{Worked-Out PDEs}
\newcommand{\name}{Jimmy Yang}

% disable indentation
\setlength\parindent{0pt}

\begin{document}

\begin{flushright}
\name
\end{flushright}

\begin{center}
\Large
\bfseries
\doctitle
\end{center}

\section{Electric potential inside long rectangular tube}

Problem statement:

\begin{displayquote}
Within a very long (extends in the z-direction), rectangular, hollow pipe, there are no electric charges. It has width $a$ and height $b$. The walls of this pipe are kept at a known voltage. The walls at $x=0,a$ are kept at constant potential $V_1$. Meanwhile, the walls at $y=\pm b/2$ are kept at constant potential $V_2$. Find potential inside the pipe.
\end{displayquote}

Solution: \\

We need to solve the 2D Laplacian $\nabla^2 V (x, y)=0$ with the following boundary conditions:

\begin{enumerate}[i]
  \item $V(0,y)=V(a,y)=V_1$
  \item $V(x,b/2)=V(x,-b/2)=V_2$
\end{enumerate}

One way to solve this is to break into the following two subproblems:

\begin{enumerate}
  \item Solve $\nabla^2 V=0$ with boundary conditions: $V(0,y)=V(a,y)=V_1$ and $V(x,b/2)=V(x,-b/2)=0$.
  \item Solve $\nabla^2 V=0$ with boundary conditions: $V(0,y)=V(a,y)=0$ and $V(x,b/2)=V(x,-b/2)=V_2$.
\end{enumerate}

But we can be more clever by changing things a bit:

\begin{enumerate}
  \item Solve $\nabla^2 V=0$ with boundary conditions: $V(0,y)=V(a,y)=V_1-V_2$ and $V(x,b/2)=V(x,-b/2)=0$.
  \item Solve $\nabla^2 V=0$ with boundary conditions: $V(0,y)=V(a,y)=V_2$ and $V(x,b/2)=V(x,-b/2)=V_2$.
\end{enumerate}

\subsection*{Subproblem 1}

By separating variables using $V(x,y)=f(x)g(y)$:

\begin{gather}
\nabla^2 V=\frac{\partial^2 V}{\partial x^2}+\frac{\partial^2 V}{\partial y^2}=0 \\
f''(x)g(y)+f(x)g''(y)=0 \\
\frac{f''(x)}{f(x)}+\frac{g''(y)}{g(y)}=0 \\
C_1+C_2=0
\end{gather}

The assignment of constants will depend on boundary conditions. One of them will be positive and another will be negative. In order to satisfy the symmetric boundary condition $V(0,y)=V(a,y)=V_1-V_2$, we will use hyperbolics for x-equation ($C_1=k^2$). On the other hand, we will use sinusoidals for y-equation ($C_1=-k^2$) to satisfy the vanishing boundary conditions $V(x,b/2)=V(x,-b/2)=0$. The respective eigenvalue problems are:

\begin{gather}
f''(x)=k^2 f(x) \\
g''(y)=-k^2g(y)
\end{gather}

As usual, the general solutions to these ODEs are:

\begin{gather}
f(x)=A\cosh(kx)+B\sinh(kx) \\
g(y)=C\cos(ky)+D\sin(ky)
\end{gather}

\noindent
The boundary conditions demand that we need to shift coordinates:

\begin{gather}
f(x)=A\cosh\left(k\left(x-\frac{a}{2}\right)\right) \\
g(y)=C\cos\left(k\left(y+\frac{b}{2}\right)\right)+D\sin\left(k\left(y+\frac{b}{2}\right)\right)
\end{gather}

Note that in $f(x)$, the constant $B$ is set to zero since $\sinh$ is an odd function which is not suitable for even symmetry. \\

The boundary conditions for y-equation are $g(b/2)=g(-b/2)=0$:

\begin{gather}
g\left(-\frac{b}{2}\right)=C\cos(0)+D\sin(0)=C=0 \\
g\left(\frac{b}{2}\right)=D\sin\left(kb\right)=0 \\
kb=n\pi \\
k=\frac{n\pi}{b}
\end{gather}

For $n=1,2,3,\dots$. The solutions to ODEs are:

\begin{gather}
f_n(x)=A_n\cosh\left(\frac{n\pi}{b}\left(x-\frac{a}{2}\right)\right) \\
g_n(y)=D_n\sin\left(\frac{n\pi}{b}\left(y+\frac{b}{2}\right)\right)
\end{gather}

The general solution is just a linear combination of $f_n (x) g_n (y)$:

\begin{gather}
V(x,y)=\sum_{n=1}^\infty f_n (x) g_n (y) = \sum_{n=1}^\infty C_n \cosh\left(\frac{n\pi}{b}\left(x-\frac{a}{2}\right)\right) \sin\left(\frac{n\pi}{b}\left(y+\frac{b}{2}\right)\right)
\end{gather}

We define the following inner product:

\begin{gather}
\langle f(y), g(y) \rangle = \int_{-\frac{b}{2}}^{\frac{b}{2}} f(y) g(y) dy
\end{gather}

Applying the last boundary condition $V(0,y)=V(a,y)=V_1-V_2$ and inner product:

\begin{gather}
V(0,y) = V(a,y) = \sum_{n=1}^\infty C_n \cosh\left(\frac{n\pi a}{2b}\right) \sin\left(\frac{n\pi}{b}\left(y+\frac{b}{2}\right)\right)=V_1-V_2 \\
\left\langle \sin\left(\frac{m\pi}{b}\left(y+\frac{b}{2}\right)\right), \sum_{n=1}^\infty C_n \cosh\left(\frac{n\pi a}{2b}\right) \sin\left(\frac{n\pi}{b}\left(y+\frac{b}{2}\right)\right) \right\rangle = \left\langle \sin\left(\frac{m\pi}{b}\left(y+\frac{b}{2}\right)\right), V_1-V_2 \right\rangle \\
\sum_{n=1}^\infty C_n \cosh\left(\frac{n\pi a}{2b}\right) \left\langle \sin\left(\frac{m\pi}{b}\left(y+\frac{b}{2}\right)\right), \sin\left(\frac{n\pi}{b}\left(y+\frac{b}{2}\right)\right) \right\rangle = \left\langle \sin\left(\frac{m\pi}{b}\left(y+\frac{b}{2}\right)\right), V_1-V_2 \right\rangle
\end{gather}

But notice that:

\begin{gather}
\int_{-\frac{b}{2}}^{\frac{b}{2}} \sin\left(\frac{m\pi}{b}\left(y+\frac{b}{2}\right)\right) \sin\left(\frac{n\pi}{b}\left(y+\frac{b}{2}\right)\right) dy = \begin{cases} 0 & m \ne n \\ \frac{b}{2} & m=n \end{cases}
\end{gather}

Applying orthogonality gives:

\begin{gather}
C_n \cosh\left(\frac{n\pi a}{2b}\right) \left\langle \sin\left(\frac{n\pi}{b}\left(y+\frac{b}{2}\right)\right), \sin\left(\frac{n\pi}{b}\left(y+\frac{b}{2}\right)\right) \right\rangle = \left\langle \sin\left(\frac{n\pi}{b}\left(y+\frac{b}{2}\right)\right), V_1-V_2 \right\rangle \\
C_n \cosh\left(\frac{n\pi a}{2b}\right) = \frac{2}{b} (V_1-V_2) \left\langle \sin\left(\frac{n\pi}{b}\left(y+\frac{b}{2}\right)\right), 1 \right\rangle \\
C_n \cosh\left(\frac{n\pi a}{2b}\right) = \begin{cases} \frac{4}{n\pi} (V_1-V_2) & \text{odd n} \\ 0 & \text{even n} \end{cases}
\end{gather}

Now the solution for subproblem 1 is:

\begin{gather}
V_{\text{s1}}(x,y) = \sum_{n=1,\text{odd}}^\infty \frac{4}{n\pi} (V_1-V_2) \frac{\cosh\left(\frac{n\pi}{b}\left(x-\frac{a}{2}\right)\right)}{\cosh\left(\frac{n\pi a}{2b}\right)} \sin\left(\frac{n\pi}{b}\left(y+\frac{b}{2}\right)\right)
\end{gather}

\subsection*{Subproblem 2}

This one is easy. It is a constant:

\begin{gather}
V_{\text{s2}}(x,y) = V_2
\end{gather}

\subsection*{Solution}

The final solution is just $V=V_{\text{s1}}+V_{\text{s2}}$:

\begin{gather}
V(x,y) = V_2 + \sum_{n=1,\text{odd}}^\infty \frac{4}{n\pi} (V_1-V_2) \frac{\cosh\left(\frac{n\pi}{b}\left(x-\frac{a}{2}\right)\right)}{\cosh\left(\frac{n\pi a}{2b}\right)} \sin\left(\frac{n\pi}{b}\left(y+\frac{b}{2}\right)\right)
\end{gather}

\section{Electric potential inside and outside charged sphere}

Problem statement:

\begin{displayquote}
A hollow sphere of radius $R$ has a surface charge density $\sigma = \sigma_0\sin^2\theta$, where $\sigma_0$ is a constant. Find potential inside and outside the sphere.
\end{displayquote}

Solution: \\

We need to solve the Laplacian $\nabla^2 V(r, \theta)=0$ with the following boundary conditions:

\begin{enumerate}[i]
  \item $V_{in}(r=0)=\text{finite}$
  \item $V_{out}(r \rightarrow \infty) = 0$
  \item $V_{in} (r=R) = V_{out} (r=R)$
  \item $\displaystyle \frac{\partial V_{out}}{\partial r}\Bigr|_{\substack{r=R}}-\frac{\partial V_{in}}{\partial r}\Bigr|_{\substack{r=R}}=-\frac{\sigma_0\sin^2\theta}{\epsilon_0}$
\end{enumerate}

The general solution for spherical Laplacian, assuming azimuthal symmetry, is:

\begin{gather}
V (r, \theta) = \sum_{l=0}^\infty \left( A_l r^l + \frac{B_l}{r^{l+1}} \right) P_l (\cos\theta)
\end{gather}

where $P_l$ are Legendre polynomials of order $l$. \\

From boundary conditions i and ii, $V_{in}$ and $V_{out}$ will be:

\begin{gather}
V_{in} (r, \theta) = \sum_{l=0}^\infty A_l r^l P_l (\cos\theta) \\
V_{out} (r, \theta) = \sum_{l=0}^\infty \frac{B_l}{r^{l+1}} P_l (\cos\theta)
\end{gather}

Boundary condition iii gives:

\begin{gather}
\sum_{l=0}^\infty A_l R^l P_l (\cos\theta) = \sum_{l=0}^\infty \frac{B_l}{R^{l+1}} P_l (\cos\theta)
\end{gather}

Boundary condition iv gives:

\begin{gather}
\sum_{l=0}^\infty -(l+1)\frac{B_l}{R^{l+2}} P_l (\cos\theta) - \sum_{l=0}^\infty l A_l R^{l-1} P_l (\cos\theta) = -\frac{\sigma_0}{\epsilon_0} \sin^2\theta \\
\sum_{l=0}^\infty \left( -(l+1)\frac{B_l}{R^{l+2}} - l A_l R^{l-1} \right) P_l (\cos\theta) = -\frac{\sigma_0}{\epsilon_0} (1-\cos^2\theta) = -\frac{\sigma_0}{\epsilon_0} \left(\frac{2}{3}-\frac{2}{3}P_2(\cos\theta)\right)
\end{gather}

It seems like only polynomials of degrees $l=0,2$ matter, so from equations 31 and 32:

\begin{gather}
A_0 + A_2 R^2 P_2 (\cos\theta) = \frac{B_0}{R} + \frac{B_2}{R^3} P_2 (\cos\theta) \\
-\frac{B_0}{R^{2}} + \left( -3\frac{B_2}{R^{4}} - 2 A_2 R \right) P_2 (\cos\theta) = -\frac{2}{3} \frac{\sigma_0}{\epsilon_0} + \frac{2}{3} \frac{\sigma_0}{\epsilon_0} P_2(\cos\theta)
\end{gather}

Matching coefficients gives the following system to solve for constants $A_0, A_2, B_0, B_2$:

\begin{gather}
A_0 = \frac{B_0}{R} \\
A_2 R^2 = \frac{B_2}{R^3} \\
\frac{B_0}{R^{2}} = \frac{2}{3} \frac{\sigma_0}{\epsilon_0} \\
-3\frac{B_2}{R^{4}} - 2 A_2 R = \frac{2}{3} \frac{\sigma_0}{\epsilon_0}
\end{gather}

After a painful bit of algebra, the constants are:

\begin{gather}
A_0 = \frac{2}{3}\frac{\sigma_0}{\epsilon_0} R \\
A_2 = -\frac{2}{15} \frac{\sigma_0}{\epsilon_0} \frac{1}{R} \\
B_0 = \frac{2}{3}\frac{\sigma_0}{\epsilon_0} R^2 \\
B_2 = -\frac{2}{15} \frac{\sigma_0}{\epsilon_0} R^4
\end{gather}

The final solution is:

\begin{gather}
V_{in} (r, \theta) = \frac{2}{3}\frac{\sigma_0}{\epsilon_0} R - \frac{2}{15} \frac{\sigma_0}{\epsilon_0} \frac{r^2}{R} P_2 (\cos\theta) \\
V_{out} (r, \theta) = \frac{2}{3}\frac{\sigma_0}{\epsilon_0} \frac{R^2}{r} - \frac{2}{15} \frac{\sigma_0}{\epsilon_0} \frac{R^4}{r^3} P_2 (\cos\theta)
\end{gather}

\section{Electric potential for long cylinder}

Problem statement:

\begin{displayquote}
Two half-cylinders of radius $R$ made out of conducting material are separated by a thin air gap. The left half cylinder ($\pi/2 < \phi < 3\pi/2$) is held at potential $-V_0$ and the right half-cylinder ($0 < \phi < \pi/2$ or $3\pi/2 < \phi < 2\pi$) at potential $+V_0$. Find potential inside and outside the cylinder.
\end{displayquote}

Solution: \\

We need to solve the Laplacian $\nabla^2 V(s, \phi)=0$ for a cylinder with the following boundary conditions:

\begin{enumerate}[i]
  \item $V_{in}(s=0)=\text{finite}$
  \item $V_{out}(s \rightarrow \infty) = 0$
  \item $V_{in} (s=R) = V_{out} (s=R)$
  \item $V(s=R)=\begin{cases} V_0 & 0 < \phi < \pi/2 \\ -V_0 & \pi/2 < \phi < 3\pi/2 \\ V_0 & 3\pi/2 < \phi < 2\pi \end{cases}$
\end{enumerate}

The general solution for the cylindrical Laplacian is:

\begin{gather}
V (s, \phi) = A + B \ln s + \sum_{n=1}^\infty \left( A_n s^n + B_n s^{-n} \right) \left( C_n \cos (n\phi) + D_n \sin (n\phi) \right)
\end{gather}

To satisfy boundary conditions i and ii, $V_{in}$ and $V_{out}$ will be:

\begin{gather}
V_{in} (s, \phi) = \sum_{n=1}^\infty A_n s^n \cos (n\phi) \\
V_{out} (s, \phi) = \sum_{n=1}^\infty B_n s^{-n} \cos (n\phi)
\end{gather}

Note that only $\cos$ terms are used because boundary condition iv demands that $V$ be negative when $\pi/2 < \phi < 3\pi/2$ which is satisfied by $\cos$ function. \\

We define the following inner product:

\begin{gather}
\langle f(\phi), g(\phi) \rangle = \int_0^{2\pi} f(\phi) g(\phi) d\phi
\end{gather}

Boundary condition iv will be used to find constants $A_n$ and $B_n$. We will start with $V_{in}$:

\begin{gather}
V_{in} (R, \phi) = \sum_{n=1}^\infty A_n R^n \cos (n\phi) = \begin{cases} V_0 & 0 < \phi < \pi/2 \\ -V_0 & \pi/2 < \phi < 3\pi/2 \\ V_0 & 3\pi/2 < \phi < 2\pi \end{cases}
\end{gather}

Applying the inner product to both sides gives:

\begin{gather}
\left\langle \cos(m\phi), \sum_{n=1}^\infty A_n R^n \cos (n\phi) \right\rangle = \left\langle \cos(m\phi), \begin{cases} V_0 & 0 < \phi < \pi/2 \\ -V_0 & \pi/2 < \phi < 3\pi/2 \\ V_0 & 3\pi/2 < \phi < 2\pi \end{cases} \right\rangle \\
\sum_{n=1}^\infty A_n R^n \left\langle \cos(n\phi), \cos (m\phi) \right\rangle = \left\langle \cos(m\phi), \begin{cases} V_0 & 0 < \phi < \pi/2 \\ -V_0 & \pi/2 < \phi < 3\pi/2 \\ V_0 & 3\pi/2 < \phi < 2\pi \end{cases} \right\rangle
\end{gather}

However notice that:

\begin{gather}
\left\langle \cos(n\phi), \cos (m\phi) \right\rangle = \int_0^{2\pi}\cos(n\phi) \cos (m\phi) d\phi = \begin{cases} 0 & m \ne n \\ \pi & m=n \end{cases}
\end{gather}

Applying the orthogonality of $\cos(n\phi)$ to equation 7:

\begin{gather}
A_n R^n \left\langle \cos(n\phi), \cos (n\phi) \right\rangle = \left\langle \cos(n\phi), \begin{cases} V_0 & 0 < \phi < \pi/2 \\ -V_0 & \pi/2 < \phi < 3\pi/2 \\ V_0 & 3\pi/2 < \phi < 2\pi \end{cases} \right\rangle \\
A_n R^n \pi = \int_0^{\frac{\pi}{2}} V_0 \cos(n\phi) d\phi - \int_{\frac{\pi}{2}}^{\frac{3\pi}{2}} V_0 \cos(n\phi) d\phi + \int_{\frac{3\pi}{2}}^{2\pi} V_0 \cos(n\phi) d\phi \\
A_n R^n = \frac{V_0}{\pi} \left[ \int_0^{\frac{\pi}{2}} \cos(n\phi) d\phi - \int_{\frac{\pi}{2}}^{\frac{3\pi}{2}} \cos(n\phi) d\phi + \int_{\frac{3\pi}{2}}^{2\pi} \cos(n\phi) d\phi \right] \\
A_n R^n = \frac{2V_0}{n\pi} \left[ \sin\left( \frac{n\pi}{2} \right) - \sin\left( \frac{3n\pi}{2} \right) \right] \\
A_n R^n = \begin{cases} \frac{4V_0}{n\pi} (-1)^{\frac{n-1}{2}} & \text{odd n} \\ 0 & \text{even n} \end{cases}
\end{gather}

Calculating constant $B_n$ is similar:

\begin{gather}
B_n R^{-n} = \begin{cases} \frac{4V_0}{n\pi} (-1)^{\frac{n-1}{2}} & \text{odd n} \\ 0 & \text{even n} \end{cases}
\end{gather}

The final solutions are now therefore:

\begin{gather}
V_{in} (s, \phi) = \sum_{n=1,\text{odd}}^\infty \frac{4V_0}{n\pi} (-1)^{\frac{n-1}{2}} \left(\frac{s}{R}\right)^n \cos (n\phi) \\
V_{out} (s, \phi) = \sum_{n=1,\text{odd}}^\infty \frac{4V_0}{n\pi} (-1)^{\frac{n-1}{2}} \left(\frac{R}{s}\right)^n \cos (n\phi)
\end{gather}

\end{document}